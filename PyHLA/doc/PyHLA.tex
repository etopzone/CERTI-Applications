\documentclass{howto}

\title{Using M\&S HLA in Python}
\release{0.1}

\author{Petr Gotthard}
\authoraddress{petr.gotthard@centrum.cz}

\begin{document}
\maketitle

% This makes the Abstract go on a separate page in the HTML version;
% if a copyright notice is used, it should go immediately after this.
%
\ifhtml
\chapter*{Front Matter\label{front}}
\fi

% Copyright statement should go here, if needed.
% ...

% The abstract should be a paragraph or two long.
\begin{abstract}
\noindent
This document describes Python language binding for the Modeling\&Simulation
High Level Architecture (M\&S HLA).
The M\&S HLA is a general purpose architecture for distributed computer
simulation systems. It's defined in [HLA1.3] and [IEEE 1516.1].

The \module{PyHLA} module provides Python wrapper for the C++ HLA API, and
\method{pack}/\method{unpack} methods for IEEE 1516.2 data types.
\end{abstract}

\tableofcontents

\section{Introduction}

The \module{PyHLA} module provides Python wrapper for the C++ HLA API, and
\method{pack}/\method{unpack} methods for IEEE 1516.2 data types.

The M\&S HLA is a general purpose architecture for distributed computer
simulation systems. It's defined in [HLA1.3] and [IEEE 1516.1].

\subsection{Motivation}

The \module{PyHLA} module aims to enable rapid development of HLA federates,
i.e. to simplify the activity 4.3 of FEDEP [IEEE 1516.3].

The HLA standard defines C++ mappings [IEEE 1516.1], but a significant
development effort is necessary to develop a HLA compliant federate in C++.
The development effort can be reduced by using
\begin{enumerate}
\item higher level interfaces, e.g. the Protocol Independent Interface in M\"{A}K VR-Link
\item code generators, e.g. the GENESIS developed by ONERA [GENESIS]
\end{enumerate}

The HLA standard does not cover all aspects.
The C++ API for value encoding [IEEE 1516.2] is not standardized. Every HLA
developer needs to implement the value encoding functions.

Integrating HLA into the Python language may reduce the development and
maintenance effort (compared to C/C++).

The Python language provides several benefits
\begin{enumerate}
\item Dynamic data types. High-level programming language.
\item Interpreted. Easy modifications.
\item Powerful plug-in system.
\item Many plug-ins providing scientific calculations, or geodetic conversions.
\end{enumerate}

\subsection{Installation Instructions}

\paragraph{Windows}

A binary installer is provided for the latest CERTI and Python release.

The wizzard will guide you through the installation process. The \module{PyHLA}
module should be installed into the Python \file{site-packages} directory.
When prompted for a destination folder, you should enter something like
\file{C:/Program Files/Python2.5/Lib/site-packages}.

\paragraph{Linux}

The \module{PyHLA} module requires
\begin{itemize}
\item Classic Python interpreter (\ulink{CPython}{http://www.python.org/download})
\item HLA1.3 RTI with C++ API (\ulink{CERTI}{http://www.cert.fr/CERTI}, \ulink{M\"{A}K RTI}{http://www.mak.com/products/rti.php})
\end{itemize}

The \module{PyHLA} source codes can be obtained from the \ulink{CERTI}{https://savannah.nongnu.org/projects/certi}
applications repository.
\begin{verbatim}
cvs -d:pserver:anonymous@cvs.sv.gnu.org:/sources/certi co applications/PyHLA
\end{verbatim}

To build the \module{PyHLA} module you must provide a path to the Python
\file{site-packages} directory.

If no \makevar{CMAKE_INSTALL_PREFIX} is provided, the path is determined
automatically. The source codes can be compiled and installed using
\begin{verbatim}
cmake .
make install
\end{verbatim}

Alternatively, you may provide the path to the \file{site-packages} directory
in the \makevar{CMAKE_INSTALL_PREFIX} cmake variable.
\begin{verbatim}
cmake . -DCMAKE_INSTALL_PREFIX=/usr/lib/python2.5/site-packages
make install
\end{verbatim}

\subsection{Performance}

Discuss the performance penalty. Use the DFSS/DMAIC methodology.

Design experiments for performance measurements.

Measure performance of Python and plain C++ federates. Use multiple RTI: CERTI and MÄK RTI.

Discuss the results.

\section{Value Encoding}

The \module{hla.omt} module provides
\begin{itemize}
\item Automatic IEEE1516.2 value encoding (pack and unpack functions).
\item Direct import of OMT DIF files.
\end{itemize}

First, import the module functions
\begin{verbatim}
import hla.omt as fom
\end{verbatim}

Each encoding object provides the \method{pack}, \method{unpack} methods, and
the \member{octetBoundary} property.

\begin{classdesc}{HLAbasic}{}
\begin{methoddesc}[HLAbasic]{pack}{value}
Returns encoded value.
\begin{verbatim}
encodedValue = fom.HLAfloat64LE.pack(3.14)
encodedValue += fom.HLAinteger16BE.pack(42)
\end{verbatim}
\end{methoddesc}
\begin{methoddesc}[HLAbasic]{unpack}{value}
Returns a (value, offset) tuple. The offset indicates number of consumed
bytes, i.e. offset of the next unpack operation.
\begin{verbatim}
value1, offset2 = fom.HLAfloat64LE.unpack(encodedValue)
value2, offset3 = fom.HLAinteger16BE.unpack(encodedValue[offset2:])
print value1, value2
\end{verbatim}
\end{methoddesc}
\begin{memberdesc}[int]{octetBoundary}
Read-only property. The octet boundary (in bytes) of this data type.
\end{memberdesc}
\end{classdesc}

The \module{hla.omt} module defines the following basic types.
\begin{tableiii}{l|l|l}{constant}{type name}{bytes}{endian}
\lineiii{HLAinteger16BE}{2}{Big}
\lineiii{HLAinteger32BE}{4}{Big}
\lineiii{HLAinteger64BE}{8}{Big}
\lineiii{HLAfloat32BE}{4}{Big}
\lineiii{HLAfloat64BE}{8}{Big}
\lineiii{HLAinteger16LE}{2}{Little}
\lineiii{HLAinteger32LE}{4}{Little}
\lineiii{HLAinteger64LE}{8}{Little}
\lineiii{HLAfloat32LE}{4}{Little}
\lineiii{HLAfloat64LE}{8}{Little}
\end{tableiii}

Additional data types can be defined using the \class{HLAencoding},
\class{HLAfixedArray}, \class{HLAvariableArray}, or \class{HLAfixedRecord}
class, or imported from an OMT DIF file.

\begin{funcdesc}{HLAuse}{filename}
Imports data types from an OMT DIF file. Format of the OMT DIF file is
defined in IEEE 1516.2.

All the RPR-FOM classes and data types will become accessible from the
module's dictionary.

\begin{verbatim}
fom.HLAuse('rpr2-d18.xml')

encodedValue = fom.WorldLocationStruct.pack({'X':0, 'Y':3.12, 'Z':1})
value, offset = fom.WorldLocationStruct.unpack(encodedValue)
\end{verbatim}

\end{funcdesc}

\begin{classdesc}{HLAencoding}{typename}
Facilitates a forward type declaration. The typename is resolved from the
module's dictionary once the \method{pack} or \method{unpack} method is called.

\begin{verbatim}
newType = fom.HLAencoding('WorldLocationStruct')
\end{verbatim}
is equivalent to the following OMT DIF
\begin{verbatim}
<simpleData name="newType" representation="WorldLocationStruct"/>
\end{verbatim}
\end{classdesc}

\begin{classdesc}{HLAenumerated}{name, representation, \{enumeratorName:value\}}
Defines an enumeration.

\begin{verbatim}
HLAboolean = fom.HLAenumerated("HLAboolean",
    HLAinteger32BE, {"HLAfalse":0, "HLAtrue":1})
\end{verbatim}
is equivalent to the following OMT DIF
\begin{verbatim}
<enumeratedData name="HLAboolean" representation="HLAinteger32BE">
    <enumerator name="HLAfalse" values="0"/>
    <enumerator name="HLAtrue" values="1"/>
</enumeratedData>
\end{verbatim}

The enumerators are accessible via attributes. The values behave like integers,
but don't support arithmetic.
\begin{verbatim}
print HLAboolean.HLAtrue
\end{verbatim}

The \method{pack} function takes an enumerator, or an integer value.
\begin{verbatim}
encodedValue = fom.HLAboolean.pack(HLAboolean.HLAtrue)
\end{verbatim}
\end{classdesc}

\begin{classdesc}{HLAfixedArray}{name, elementType, cardinality}
Defines a fixed-length array.

\begin{verbatim}
arrayType = fom.HLAfixedArray("arrayType", fom.HLAfloat32LE, 3)
\end{verbatim}
is equivalent to the following OMT DIF
\begin{verbatim}
<arrayData name="arrayType"
    encoding="HLAfixedArray" dataType="HLAfloat32LE" cardinality="3"/>
\end{verbatim}

The \method{pack} function takes a sequence object.
\begin{verbatim}
encodedValue = fom.arrayType.pack([3.14, 17, 42])
\end{verbatim}
\end{classdesc}

\begin{classdesc}{HLAvariableArray}{name, elementType, cardinality = None}
Defines a variable-length array.

Dynamic cardinality is represented by the \class{None} value.

\begin{verbatim}
newType = fom.HLAvariableArray("newType", fom.HLAinteger16BE, None)
\end{verbatim}
is equivalent to the following OMT DIF
\begin{verbatim}
<arrayData name="newType"
    encoding="HLAvariableArray" dataType="HLAinteger16BE" cardinality="Dynamic"/>
\end{verbatim}
\end{classdesc}

\begin{classdesc}{HLAfixedRecord}{name, [(fieldName, fieldType)]}
Defines a fixed record.

\begin{verbatim}
recordType = fom.HLAfixedRecord("recordType",
    [('X',fom.HLAinteger16BE), ('Y',fom.HLAfloat32BE)])
\end{verbatim}
is equivalent to the following OMT DIF
\begin{verbatim}
<fixedRecordData name="recordType">
    <field name="X" dataType="HLAinteger16BE"/>
    <field name="Y" dataType="HLAfloat32BE"/>
</fixedRecordData>
\end{verbatim}

The \method{pack} function takes a mapping object.
\begin{verbatim}
encodedValue = fom.recordType.pack({'X':0, 'Y':3.12})
\end{verbatim}
\end{classdesc}

\begin{classdesc}{HLAvariantRecord}{name, (discriminant, dataType), [([enumerators], altName, altType)]}
Defines a variant record.

\begin{verbatim}
recordType = fom.HLAvariantRecord("recordType", ("Axis",AxisEnum),
    [(["TX"],"X",HLAinteger32LE), (["TY",None],"Y",HLAinteger32LE)])
\end{verbatim}
is equivalent to the following OMT DIF
\begin{verbatim}
<variantRecordData name="recordType"
    encoding="HLAvariantRecord" dataType="AxisEnum" discriminant="Axis">
    <alternative enumerator="TX" name="X" dataType="HLAinteger32LE"/>
    <alternative enumerator="TY,HLAother" name="Y" dataType="HLAinteger32LE"/>
</variantRecordData>
\end{verbatim}

The \constant{HLAother} enumerator is represented by the \class{None} value.

The \method{pack} function takes a mapping object.
\begin{verbatim}
encodedValue = fom.recordType.pack({'Axis':AxisEnum.TX, 'X':7.2})
\end{verbatim}
\end{classdesc}

% $Id: module.tex,v 1.3 2008/10/13 17:15:35 gotthardp Exp $

\section{Value Encoding}

The \module{hla.omt} module provides
\begin{itemize}
\item Automatic IEEE1516.2 value encoding (pack and unpack functions).
\item Direct import of OMT DIF files.
\end{itemize}

First, import the module functions
\begin{verbatim}
import hla.omt as fom
\end{verbatim}

Each encoding object provides the \method{pack}, \method{unpack} methods, and
the \member{octetBoundary} property.

\begin{classdesc}{HLAbasic}{}
\begin{methoddesc}[HLAbasic]{pack}{value}
Returns encoded value.
\begin{verbatim}
encodedValue = fom.HLAfloat64LE.pack(3.14)
encodedValue += fom.HLAinteger16BE.pack(42)
\end{verbatim}
\end{methoddesc}
\begin{methoddesc}[HLAbasic]{unpack}{value}
Returns a (value, offset) tuple. The offset indicates number of consumed
bytes, i.e. offset of the next unpack operation.
\begin{verbatim}
value1, offset2 = fom.HLAfloat64LE.unpack(encodedValue)
value2, offset3 = fom.HLAinteger16BE.unpack(encodedValue[offset2:])
print value1, value2
\end{verbatim}
\end{methoddesc}
\begin{memberdesc}[int]{octetBoundary}
Read-only property. The octet boundary (in bytes) of this data type.
\end{memberdesc}
\end{classdesc}

The \module{hla.omt} module defines the following basic types.
\begin{tableiii}{l|l|l}{constant}{type name}{bytes}{endian}
\lineiii{HLAinteger16BE}{2}{Big}
\lineiii{HLAinteger32BE}{4}{Big}
\lineiii{HLAinteger64BE}{8}{Big}
\lineiii{HLAfloat32BE}{4}{Big}
\lineiii{HLAfloat64BE}{8}{Big}
\lineiii{HLAinteger16LE}{2}{Little}
\lineiii{HLAinteger32LE}{4}{Little}
\lineiii{HLAinteger64LE}{8}{Little}
\lineiii{HLAfloat32LE}{4}{Little}
\lineiii{HLAfloat64LE}{8}{Little}
\end{tableiii}

Additional data types can be defined using the \class{HLAencoding},
\class{HLAfixedArray}, \class{HLAvariableArray}, or \class{HLAfixedRecord}
class, or imported from an OMT DIF file.

\begin{funcdesc}{HLAuse}{filename}
Imports data types from an OMT DIF file. Format of the OMT DIF file is
defined in IEEE 1516.2.

All the RPR-FOM classes and data types will become accessible from the
module's dictionary.

\begin{verbatim}
fom.HLAuse('rpr2-d18.xml')

encodedValue = fom.WorldLocationStruct.pack({'X':0, 'Y':3.12, 'Z':1})
value, offset = fom.WorldLocationStruct.unpack(encodedValue)
\end{verbatim}

\end{funcdesc}

\begin{classdesc}{HLAencoding}{typename}
Facilitates a forward type declaration. The typename is resolved from the
module's dictionary once the \method{pack} or \method{unpack} method is called.

\begin{verbatim}
newType = fom.HLAencoding('WorldLocationStruct')
\end{verbatim}
is equivalent to the following OMT DIF
\begin{verbatim}
<simpleData name="newType" representation="WorldLocationStruct"/>
\end{verbatim}
\end{classdesc}

\begin{classdesc}{HLAenumerated}{name, representation, \{enumeratorName:value\}}
Defines an enumeration.

\begin{verbatim}
HLAboolean = fom.HLAenumerated("HLAboolean",
    HLAinteger32BE, {"HLAfalse":0, "HLAtrue":1})
\end{verbatim}
is equivalent to the following OMT DIF
\begin{verbatim}
<enumeratedData name="HLAboolean" representation="HLAinteger32BE">
    <enumerator name="HLAfalse" values="0"/>
    <enumerator name="HLAtrue" values="1"/>
</enumeratedData>
\end{verbatim}

The enumerators are accessible via attributes. The values behave like integers,
but don't support arithmetic.
\begin{verbatim}
print HLAboolean.HLAtrue
\end{verbatim}

The \method{pack} function takes an enumerator, or an integer value.
\begin{verbatim}
encodedValue = fom.HLAboolean.pack(HLAboolean.HLAtrue)
\end{verbatim}
\end{classdesc}

\begin{classdesc}{HLAfixedArray}{name, elementType, cardinality}
Defines a fixed-length array.

\begin{verbatim}
arrayType = fom.HLAfixedArray("arrayType", fom.HLAfloat32LE, 3)
\end{verbatim}
is equivalent to the following OMT DIF
\begin{verbatim}
<arrayData name="arrayType"
    encoding="HLAfixedArray" dataType="HLAfloat32LE" cardinality="3"/>
\end{verbatim}

The \method{pack} function takes a sequence object.
\begin{verbatim}
encodedValue = fom.arrayType.pack([3.14, 17, 42])
\end{verbatim}
\end{classdesc}

\begin{classdesc}{HLAvariableArray}{name, elementType, cardinality = None}
Defines a variable-length array.

Dynamic cardinality is represented by the \class{None} value.

\begin{verbatim}
newType = fom.HLAvariableArray("newType", fom.HLAinteger16BE, None)
\end{verbatim}
is equivalent to the following OMT DIF
\begin{verbatim}
<arrayData name="newType"
    encoding="HLAvariableArray" dataType="HLAinteger16BE" cardinality="Dynamic"/>
\end{verbatim}
\end{classdesc}

\begin{classdesc}{HLAfixedRecord}{name, [(fieldName, fieldType)]}
Defines a fixed record.

\begin{verbatim}
recordType = fom.HLAfixedRecord("recordType",
    [('X',fom.HLAinteger16BE), ('Y',fom.HLAfloat32BE)])
\end{verbatim}
is equivalent to the following OMT DIF
\begin{verbatim}
<fixedRecordData name="recordType">
    <field name="X" dataType="HLAinteger16BE"/>
    <field name="Y" dataType="HLAfloat32BE"/>
</fixedRecordData>
\end{verbatim}

The \method{pack} function takes a mapping object.
\begin{verbatim}
encodedValue = fom.recordType.pack({'X':0, 'Y':3.12})
\end{verbatim}
\end{classdesc}

\begin{classdesc}{HLAvariantRecord}{name, (discriminant, dataType), [([enumerators], altName, altType)]}
Defines a variant record.

\begin{verbatim}
recordType = fom.HLAvariantRecord("recordType", ("Axis",AxisEnum),
    [(["TX"],"X",HLAinteger32LE), (["TY",None],"Y",HLAinteger32LE)])
\end{verbatim}
is equivalent to the following OMT DIF
\begin{verbatim}
<variantRecordData name="recordType"
    encoding="HLAvariantRecord" dataType="AxisEnum" discriminant="Axis">
    <alternative enumerator="TX" name="X" dataType="HLAinteger32LE"/>
    <alternative enumerator="TY,HLAother" name="Y" dataType="HLAinteger32LE"/>
</variantRecordData>
\end{verbatim}

The \constant{HLAother} enumerator is represented by the \class{None} value.

The \method{pack} function takes a mapping object.
\begin{verbatim}
encodedValue = fom.recordType.pack({'Axis':AxisEnum.TX, 'X':7.2})
\end{verbatim}
\end{classdesc}

% $Id: module.tex,v 1.3 2008/10/13 17:15:35 gotthardp Exp $


\section{High-Level API (NOT IMPLEMENTED)}
High-level interace. Shall encompass the most frequently used HLA features. May not encompass all of the features.

Do not hide the main loop from the user. Keep the tick() function in the Python script.

Automatic entity publication and discovery.

Example:
Connect to the RTIG.

connection = hla.Connection("federation", "federate", "model.fed")

Publish the aircraft.

\begin{verbatim}
connection.publishedEntities["OS-001"] = aircraft
\end{verbatim}

List world coordinates of all discovered aircrafts. By default, all discovered
entities are automatically stored in the discoveredEntities dictionary.

\begin{verbatim}
for entity in connection.discoveredEntities:
  if type(entity) == hla.Aircraft:
    print entity.WorldLocation
\end{verbatim}

Dynamic data types. Duck-typed.
Notifications received as method calls in sub-classed functions.

Example:
Define a custom Aircraft class.

\begin{verbatim}
class MyAircraft(Aircraft):
  pass
\end{verbatim}

Define a custom connection that will discover aircrafts only. Each aircraft will
be represented by the custom Aircraft class.

\begin{verbatim}
class MyConnection(Connection):
  def discoverObjectInstance(self, name, class):
    if class == hla.Aircraft:
      self.discoveredEntities[name] = MyAircraft()
\end{verbatim}

\end{document}

% $Id: PyHLA.tex,v 1.4 2008/10/03 15:28:16 gotthardp Exp $
